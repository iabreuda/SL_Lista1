\documentclass[a4paper, 12pt]{article}

\usepackage[portuges]{babel}
\usepackage[utf8]{inputenc}
\usepackage{amsmath}
\usepackage{indentfirst}
\usepackage{blindtext}
\usepackage{graphicx}
\usepackage[hidelinks]{hyperref}
\usepackage{gensymb}
\usepackage{pgfplots}

\author{Igor Abreu da Silva}

\title{Trabalho Final de Sistemas Lineares I}

\begin{document}

    \begin{titlepage}
        \begin{center}
            \huge{Universidade Federal do Rio de Janeiro}
            \vspace{95pt}

            \large{Lista I - Sistemas Lineares I}
            \vspace{160pt}
        \end{center}

        \begin{flushleft}
            \begin{tabbing}
                Alunos\qquad\qquad\= Igor Abreu da Silva\\
                DRE\> 112053874 \\
                Curso\> Engenharia Eletrônica \\
                Turma\> 2016/2 \\
                Professor\> Natanael Nunes de Moura Junior \\

            \end{tabbing}

        \end{flushleft}

        \begin{center}
            \vspace{\fill}
            Rio de Janeiro, 16 de Setembro de 2016
        \end{center}
    \end{titlepage}

    \newpage
    \tableofcontents
    \listoffigures
    \thispagestyle{empty}
    \newpage
    \pagenumbering{arabic}

    \section{Quest\~{a}o 1 - Conhecimentos Básicos}
        \subsection{Item a}
            \begin{figure}[!ht]
                \centering
                \includegraphics{img/Figura1.png}
                \caption{Sinais utilizados no Item A}
            \end{figure}
            Analisando os resultados, percebe-se que a inversão ou o deslocamento não alteram a energia do sinal, entretanto, a multiplicação por um fator k altera o sinal em $k^{2}$.
            \subsubsection{Sinal (a)}
            $\int_{0}^{2} 1^{2}dx + \int_{2}^{3} -1^{2}dx \Rightarrow  \int_{0}^{2}dx + \int_{2}^{3}dx = 3$
            \subsubsection{Sinal (b)}
            $\int_{0}^{2} -1^{2}dx + \int_{2}^{3} 1^{2}dx \Rightarrow  \int_{0}^{2}dx + \int_{2}^{3}dx = 3$
            \subsubsection{Sinal (c)}
            $\int_{3}^{5} 1^{2}dx + \int_{5}^{6} -1^{2}dx \Rightarrow  \int_{3}^{5}dx + \int_{5}^{5}dx = 3$
            \subsubsection{Sinal (d)}
            $\int_{0}^{2} 2^{2}dx + \int_{2}^{3} -2^{2}dx \Rightarrow  \int_{0}^{2}4dx + \int_{2}^{3}4dx = 12$
            \newpage
        \subsection{Item b}
            \begin{figure}[!ht]
                \centering
                \includegraphics{img/Figura2.png}
                \caption{Sinais utilizados no Item B}
            \end{figure}
            Repete-se o que ocorre no Item(a)
            \subsubsection{Sinal (a)}
            $\int_{0}^{1} x^{2}dx  = \frac{1}{3}$
            \subsubsection{Sinal (b)}
            $\int_{-1}^{0} (-x)^{2}dx  = \frac{1}{3}$
            \subsubsection{Sinal (c)}
            $\int_{0}^{1} (-x)^{2}dx  = \frac{1}{3}$
            \subsubsection{Sinal (d)}
            $\int_{1}^{2} (x-1)^{2}dx  \Rightarrow \int_{1}^{2} (x^{2}-2x+1)dx = \frac{8}{3} - \frac{1}{3} - 4 +1 + 2 -1 = \frac{1}{3}$
            \subsubsection{Sinal (e)}
            $\int_{0}^{1} (2x)^{2}dx  = \frac{4}{3}$
        \subsection{Item c}
            \begin{figure}[!ht]
                \centering
                \includegraphics{img/Figura3.png}
                \caption{Sinais utilizados no Item C}
            \end{figure}
            Percebe-se que nos Sinais "a" e "b" a energia de x+y é igual a energia de x e y somadas, assim com, x-y é a energia de "a" e "b" subtraída, entretanto, não podemos assumir isso como verdade pois nos Sinais "c" não existe tal relação.
            \subsubsection{Sinais (a)}
            \[E_{x} = \int_{0}^{2} 1^{2}dx = 2 \]
            \[E_{y} = \int_{0}^{1} 1^{2}dx + \int_{1}^{2} -1^{2}dx \Rightarrow 1 + 1 = 2 \]
            \[E_{x+y} = \int_{0}^{1} 2^{2}dx = 4 \]
            \[E_{x-y} = \int_{1}^{2} -2^{2}dx = 4 \]
            \subsubsection{Sinais (b)}
            \[E_{x} = \int_{0}^{2\pi} sin^{2}(x)dx \Rightarrow \int_{0}^{2\pi} \frac{1-cos(2x)}{2} \Rightarrow  \frac{1}{2}\int_{0}^{2\pi} 1dx - \frac{1}{2}\int_{0}^{2\pi} cos(2x)dx = \pi + 0 = \pi\]
            \[E_{y} = \int_{0}^{2\pi} 1^{2}dx  = 2\pi \]
            \[E_{x+y} = \int_{0}^{2\pi} (sin(x) +1)^{2}dx \Rightarrow \int_{0}^{2\pi} \frac{1-cos(2x)}{2} + 2sen(x) + 1\Rightarrow\]\[ \frac{1}{2}\int_{0}^{2\pi} 1dx - \frac{1}{2}\int_{0}^{2\pi} cos(2x)dx + 2\int_{0}^{2\pi} sin(x)dx + \int_{0}^{2\pi} 1dx= \pi + 0 + 0 + 2\pi = 3\pi\]
            \[E_{x-y} = \int_{0}^{2\pi} (sin(x) -1)^{2}dx \Rightarrow \int_{0}^{2\pi} \frac{1-cos(2x)}{2} - 2sen(x) + 1\Rightarrow\]\[ \frac{1}{2}\int_{0}^{2\pi} 1dx - \frac{1}{2}\int_{0}^{2\pi} cos(2x)dx - 2\int_{0}^{2\pi} sin(x)dx + \int_{0}^{2\pi} 1dx= \pi + 0 + 0 + 2\pi = 3\pi\]
            \subsubsection{Sinais (c)}
            \[E_{x} = \int_{0}^{\pi} sin^{2}(x)dx \Rightarrow \int_{0}^{\pi} \frac{1-cos(2x)}{2} = \frac{\pi}{2} + 0 = \frac{\pi}{2} \]
            \[E_{y} = \int_{0}^{\pi} 1^{2}dx = \pi\]
            \[E_{x+y} = \int_{0}^{\pi} (sin + 1)^{2}dx \Rightarrow \int_{0}^{\pi} \frac{1-cos(2x)dx}{2} + \int_{0}^{\pi}2sin(x) + \int_{0}^{\pi}dx = \frac{\pi}{2} + 4 + \pi = \frac{3\pi}{2} + 4\]
            \[E_{x-y} = \int_{0}^{\pi} (sin - 1)^{2}dx \Rightarrow \int_{0}^{\pi} \frac{1-cos(2x)dx}{2} + \int_{0}^{\pi}-2sin(x) + \int_{0}^{\pi}dx = \frac{\pi}{2} - 4 + \pi = \frac{3\pi}{2} - 4\]
            \newpage
        \subsection{Item d}
            \begin{figure}[!ht]
                \centering
                \includegraphics{img/Figura4.png}
                \caption{Sinais utilizados no Item D}
            \end{figure}
            \[P(x) = \frac{1}{4} \int_{-2}^{2} (x^{3})^{2}dx = \frac{64}{7}\]
            Percebe-se, que a inversão do sinal não altera a potência, entretanto a multiplicação por um escalar C, altera a potência em $C^{2}$, um comportamento igual ao já provado no calculo de energia.
            \subsubsection{Sinais (a)}
            \[P(-x) = \frac{1}{4} \int_{-2}^{2} (-x^{3})^{2}dx = \frac{64}{7}\]
            \subsubsection{Sinais (b)}
            \[P(2x) = \frac{1}{4} \int_{-2}^{2} (2x^{3})^{2}dx = \frac{256}{7}\]
            \subsubsection{Sinais (c)}
            \[P(Cx) = \frac{1}{4} \int_{-2}^{2} (Cx^{3})^{2}dx = \frac{64C^{2}}{7}\]
            \newpage
        \subsection{Item e}
            \subsubsection{Sinais (a)}
                \begin{tikzpicture}
                \begin{axis}[%
                ,xlabel=$t$
                ,ylabel=$u(t)$
                ,axis x line = bottom,axis y line = left
                ,ytick={1,2}
                ,ymax=2.5 % or enlarge y limits=upper
                ]
                \addplot+[const plot, no marks, thick] coordinates {(0,0) (1,0) (2,0) (3,0) (4,0) (5,1) (6,1) (7,1)(7,0) (8,0)} node[above,pos=.57,black]{};
                \end{axis}
                \end{tikzpicture}
            \subsubsection{Sinais (b)}
                \begin{tikzpicture}
                \begin{axis}[%
                ,xlabel=$t$
                ,ylabel=$u(t)$
                ,axis x line = bottom,axis y line = left
                ,ytick={1,2}
                ,ymax=2.5 % or enlarge y limits=upper
                ]
                \addplot+[const plot, no marks, thick] coordinates {(0,0) (1,0) (2,0) (3,0) (4,0) (5,1) (6,1) (7,2)(8,2) } node[above,pos=.57,black]{};
                \end{axis}
                \end{tikzpicture}
            \subsubsection{Sinais (c)}
                \begin{tikzpicture}
                \begin{axis}[%
                ,xlabel=$t$
                ,ylabel=$u(t)$
                ,axis x line = bottom,axis y line = left
                ,ytick={1,2,3,4}
                ,ymax=4 % or enlarge y limits=upper
                ]
                \addplot+[const plot, no marks, thick] coordinates {(0,0) (1,0) (1,1)} {};
                \addplot[domain=1:2,blue]{x*x};
                \addplot+[const plot, no marks, thick,blue] coordinates {(2,4) (2,0) (3,0)} {};
                \end{axis}
                \end{tikzpicture}
            \subsubsection{Sinais (d)}
                \begin{tikzpicture}
                \begin{axis}[%
                ,xlabel=$t$
                ,ylabel=$u(t)$
                ,axis x line = bottom,axis y line = left
                ,ytick={-1,-2,0,1,2}
                ,ymax=2 % or enlarge y limits=upper
                ,ymin=-2
                ]
                \addplot+[const plot, no marks, thick] coordinates {(0,0) (1,0) (2,0) (2,-2)} {};
                \addplot[domain=2:4,blue]{(x-4)};
                \addplot+[const plot, no marks, thick,blue] coordinates {(4,0) (5,0)} {};
                \end{axis}
                \end{tikzpicture}
        \subsection{Item f}
            \subsubsection{Sinais (a)}
            Impulso unitário em sin(0) = 0
            \subsubsection{Sinais (b)}
            $\frac{2}{9}\delta(\omega)$
            \subsubsection{Sinais (c)}
            $1(cos(-60)) = \frac{1}{2}\delta(t)$
            \subsubsection{Sinais (d)}
            $\frac{sin(\frac{-\pi}{2})}{(1)^{2}+4} = \frac{-1}{5}\delta(1-t)$
            \subsubsection{Sinais (e)}
            Substituindo-se $\omega + 3$ em $\omega$, teremos: $\frac{1}{-3j + 2}\delta(\omega+3)$
            \subsubsection{Sinais (f)}
            Usando L'hopital em $\frac{sin(k\omega)}{\omega}$, temos: $kcos(k\omega)$ que com $\omega = 0$ temos: $k\delta(\omega)$
        \subsection{Item g}
            \subsubsection{Sinais (a)}
            Como o impulso é localizado em $\tau = t$, nesse caso temos $x(\tau) = x(t)$ logo, essa integral é igual a x(t).
            \subsubsection{Sinais (b)}
            Em $\delta(\tau)$ o impulso é realizado em $\tau$ = 0, sendo $\tau = 0$, temos o resultado = x(t).
            \subsubsection{Sinais (c)}
            O impulso occore em t=0 nesta cas0 temos $e^{0} = 1$.
            \subsubsection{Sinais (d)}
            O impuso ocorre em t = 0, logo $sin(3\pi) = 0$.
            \subsubsection{Sinais (e)}
            O impulso ocorre em t = -3, logo o resultado sera $e^{3}$.
            \subsubsection{Sinais (f)}
            O impulso ocorre em t = 1, logo o resultado sera $1^{3} + 4 = 5$.
            \subsubsection{Sinais (g)}
            O impulso ocorre em t = 3, logo o resultado sera $x(2-3) = x(-1)$.
            \subsubsection{Sinais (h)}
            O impulso ocorre quando t = 3, logo o resultado sera $e^{3-1}cos(-\pi) = -e^{2}$.
        \subsection{Item h}
            \subsubsection{Sinais (a)}
            \subsubsection{Sinais (b)}
            \subsubsection{Sinais (c)}
            \subsubsection{Sinais (d)}
            \subsubsection{Sinais (e)}
            \subsubsection{Sinais (f)}
        \subsection{Item i}
            \subsubsection{Sinais (a)}
            \subsubsection{Sinais (b)}
        \subsection{Item j}
            \subsubsection{Sinais (a)}
            \subsubsection{Sinais (b)}
            \subsubsection{Sinais (c)}
            \subsubsection{Sinais (d)}
            \subsubsection{Sinais (e)}
            \subsubsection{Sinais (f)}
            \subsubsection{Sinais (g)}
            \subsubsection{Sinais (h)}
        \subsection{Item k}
            \subsubsection{Sinais (a)}
            \subsubsection{Sinais (b)}
            \subsubsection{Sinais (c)}
            \subsubsection{Sinais (d)}
            \subsubsection{Sinais (e)}
            \subsubsection{Sinais (f)}
        \subsection{Item l}
        \subsection{Item m}
            \subsubsection{Sinais (a)}
            \subsubsection{Sinais (b)}
            \subsubsection{Sinais (c)}
            \subsubsection{Sinais (d)}
            \subsubsection{Sinais (e)}
            \subsubsection{Sinais (f)}
        \subsection{Item n}
    \section{Quest\~{a}o 2 - Conhecimentos Básicos}
        \subsection{Item a}
            \subsubsection{Sinais (a)}
            \subsubsection{Sinais (b)}
            \subsubsection{Sinais (c)}
            \subsubsection{Sinais (d)}
            \subsubsection{Sinais (e)}
        \subsection{Item b}
            \subsubsection{Sinais (a)}
            \subsubsection{Sinais (b)}
            \subsubsection{Sinais (c)}
        \subsection{Item c}
            \subsubsection{Sinais (a)}
            \subsubsection{Sinais (b)}
            \subsubsection{Sinais (c)}
            \subsubsection{Sinais (d)}
        \subsection{Item d}
            \subsubsection{Sinais (a)}
            \subsubsection{Sinais (b)}
            \subsubsection{Sinais (c)}
            \subsubsection{Sinais (d)}
            \subsubsection{Sinais (e)}
        \subsection{Item e}
        \subsection{Item f}
        \subsection{Item g}
            \subsubsection{Sinais (a)}
            \subsubsection{Sinais (b)}
            \subsubsection{Sinais (c)}
            \subsubsection{Sinais (d)}
            \subsubsection{Sinais (e)}
            \subsubsection{Sinais (f)}
        \subsection{Item h}
            \subsubsection{Sinais (a)}
            \subsubsection{Sinais (b)}
            \subsubsection{Sinais (c)}
            \subsubsection{Sinais (d)}
            \subsubsection{Sinais (e)}
            \subsubsection{Sinais (f)}
            \subsubsection{Sinais (g)}
    \section{Quest\~{a}o 3 - Conhecimentos Básicos}
        \subsection{Item a}
        \subsection{Item b}
        \subsection{Item c}
        \subsection{Item d}
        \subsection{Item e}
    \section{Quest\~{a}o 4 - Conhecimentos Básicos}
        \subsection{Item a}
        \subsection{Item b}
        \subsection{Item c}
    \section{Quest\~{a}o 5 - Classificação de Sinais}
    \section{Quest\~{a}o 6 - Classificação de Sistemas}
        \subsection{Item a}
        \subsection{Item b}
        \subsection{Item c}
    \section{Quest\~{a}o 7 - Classificação de Sistemas}
        \subsection{Item a}
        \subsection{Item b}
    \section{Quest\~{a}o 8 - Energia e Potência de Sinais}
    \section{Quest\~{a}o 9 - Operação com Sinais}
        \subsection{Item a}
        \subsection{Item b}
        \subsection{Item c}
        \subsection{Item d}
    \section{Quest\~{a}o 10 - Operação com Sinais}
        \subsection{Item a}
        \subsection{Item b}
\end{document}